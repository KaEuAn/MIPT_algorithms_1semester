\documentclass[12pt]{article}
\usepackage[pdftex]{graphicx}
\usepackage[utf8x]{inputenc} 
\usepackage{amsfonts}
\usepackage[left=1.5cm,right=1.5cm,top=1.5cm,bottom=1.5cm]{geometry}
\usepackage[russian]{babel}
\DeclareGraphicsExtensions{.pdf,.png,.jpg}
\title{Рекурренты-2}  
\date{28/10/2016}  
\author{Kailiak Eugene}

\begin{document}
\maketitle
Построим необходимый дек на обычном массиве. \\
В начальный момент (при создании из массива размером n) он будет устроен так: размер массива будет $3n$ элементов, первые n ячеек будут пусты, потом будут следовать n наших элементов, затем - снова n пустых ячеек. Так же будут храниться два указателя - на начало и на конец реальных данных. \\
При добавлении элемента если массив не закончился с начала или с конца, то мы просто добавляем новый элемент и сдвигаем указатель. Иначе мы заводим новый массив, который будет уже размера $3m$, где m - новый размер реальных данных, потому что элементы вставляли и вынимали. \\
При удалении элемента из конца или начала можно так же просто сдвигать указатель. Если при этом хочется поддерживать размер массива хоть какого-то приличного размера, то можно все данные переносить в новый массив каждый раз, когда реальное количество элементов будет меньше чем $\frac{1}{9}$ от размера массива.\\
Оператор [] легко реализуется, потому что, по сути, у нас хранится обычный массив, в котором у нас есть указатель на первый элемент. Можем просто прибавить количество пустых ячеек сначала и получим такой же доступ, как в массиве, который работает за $O(1)$ \\
Очевидно, что в силу симметрии выбранной реализации операции back и front будут работать одинаково ассимптотически (учётно). Тогда будем рассматривать, например, только операции front в наших доказательствах. \\ 
Проведём амортизационный анализ операции push с помощью бухгалтерского учёта: \\
Пусть при каждой операции нам дают 10 монеток. Тогда при переполнении нашего дека нам нужно сделать максимум 3n+1 операцию - перенести элементы из старого массива в новый, добавиь новый элемент. Заметим, что при этом мы возвращаемся в начальное положение, если не считать добавление одного элемента. Но перед тем, как наш массив переполнился, должно было произойти как минимум n действий ($push\_front$ или $push_back$). Тогда у нас есть $10n$ монеток, которыми мы можем заплатить за операцию перенесения в новый массив. \\
Проведём теперь амортизационный анализ операции pop c помощью метода потенциалов: \\
Пусть за каждое действие новый потенциал становится больше на единицу: $\Phi_{i+1} = \Phi_i + 1$, а после перенесения элементов в новый массив, что выполняется за $n/3$ операции, потому что столько элементов должно остаться для того, чтобы мы переносили в новый массив элементы, мы скажем, что потенциал равен 0, поэтому амортизационная стоимость i-ой операции $a_i = n/3 + 0 - m , m > \frac{2}{3} n$? ибо как минимум столько операций должно произойти. $\Phi_i = O(f), a_i = O(1)$, где f - количество операций, следовательно, средняя амортизацонная стоимость $a = O(1)$ \\
Значит, амортизационная стоимость всех операций $O(1)$



\end{document}